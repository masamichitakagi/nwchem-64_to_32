%
% $Id$
%
\label{sec:etrans}

The NWChem electron transfer (ET) module calculates the electronic coupling energy (also called the electron transfer
matrix element) between ET reactant and product states.  The electronic coupling ($V_{RP}$), activation energy ($\Delta G^{*}$),
and nuclear reorganization energy ($\lambda$) are all components of the electron transfer rate defined by Marcus' theory, which
also depends on the temperature (reference 1):

\begin{equation}
{k_{ET}}=
\frac{2\pi}{\hbar}
V_{RP}^{2}
\frac{1}{\sqrt{4\pi \lambda k_{B}T}}
\exp \left( \frac{- \Delta G^{*}}{k_{B} T} \right)
\end{equation}

The ET module utilizes the method of {\em Corresponding Orbital Transformation} to calculate $V_{RP}$.
The only input required are the names
of the files containing the open-shell (UHF) MO vectors for the ET reactant and product states ($R$ and $P$).

The basis set used in the calculation of $V_{RP}$ must be the same as the basis set used to calculate the MO vectors of
$R$ and $P$.  The magnitude of $V_{RP}$ depends on the amount of overlap between $R$ and $P$,
which is important to consider when choosing the basis set.  Diffuse functions may be
necessary to fill in the overlap, particularly when the ET distance is long.

The MO's of $R$ and $P$ must correspond to localized states.  for instance, in the reaction $A^{ -}$ $B$ $\rightarrow$ $A$ $B^{ -}$
the transferring electron is localized on A in the reactant state and is localized on B in the product state.
To verify the localization of the electron in the calculation of the vectors, carefully examine the Mulliken population
analysis.  In order to determine which orbitals are involved in the electron transfer, use the print keyword \verb+"mulliken ao"+ 
which prints the Mulliken population of each basis function.

An effective core potential (ECP) basis can be used to replace core electrons.  However, there is one caveat: the orbitals
involved in electron transfer must not be replaced with ECP's.  Since the ET orbitals are valence orbitals, this is not usually
a problem, but the user should use ECP's with care.

Suggested references are listed below.  The first two references gives a good description 
of Marcus' two-state ET model, and the appendix of the third reference details the method used
in the ET module.

\begin{enumerate}
\item R.A. Marcus, N. Sutin, Biochimica Biophysica Acta 35, 437, (1985).
\item J.R. Bolton, N. Mataga, and G. McLendon in ``Electron Transfer in Inorganic, Organic and Biological Systems"
(American Chemical Society, Washington, D.C., 1991)
\item A. Farazdel, M. Dupuis, E. Clementi, and A. Aviram, 
J.~Am.~Chem.~Soc., 112, 4206 (1990).
\end{enumerate}

\section{{\tt VECTORS} --- input of MO vectors for ET reactant and product states}
\label{sec:etransvectors}

\begin{verbatim}
  VECTORS [reactants] <string reactants_filename>
  VECTORS [products ] <string products_filename>
\end{verbatim}

In the \verb+VECTORS+ directive the user specifies the source 
of the molecular orbital vectors for the ET reactant and product states. 
This is required input, as no default filename will be set by the program.
In fact, this is the only required input in the ET module, although there are
other optional keywords described below.

\section{{\tt FOCK/NOFOCK} --- method for calculating the two-electron contribution to $V_{RP}$ }
\label{sec:etransfock}

 \begin{verbatim}
   <string (FOCK||NOFOCK) default FOCK>
 \end{verbatim}

This directive enables/disables the use of the NWChem's Fock matrix 
routine in the calculation of the two-electron portion of the ET Hamiltonian.
Since the Fock matrix routine has been optimized for speed, accuracy and parallel performance,
it is the most efficient choice.

Alternatively, the user can calculate the two-electron contribution to the ET Hamiltonian
with another subroutine which may be more accurate for systems with a small
number of basis functions, although it is slower.

\section{{\tt TOL2E} --- integral screening threshold}
\label{sec:etranstol2e}

\begin{verbatim}
  TOL2E <real tol2e default max(10e-12,min(10e-7, S(RP)*10e-7 )> 
\end{verbatim}

The variable \verb+tol2e+ is used in determining the integral
screening threshold for the evaluation of the two-electron contribution to the Hamiltonian
between the electron transfer reactant and product states.
As a default, \verb+tol2e+ is set depending on the magnitude
of the overlap between the ET reactant and product states ($S_{RP}$), and is not less than 1.0d-12 
or greater than 1.0d-7.

The input to specify the threshold explicitly within the \verb+ET+
directive is, for example:

\begin{verbatim}
  tol2e 1e-9
\end{verbatim}

\section{{\tt Example}}

The following example is for a simple electron transfer reaction, $He_{}$ $\rightarrow$ $He^{ +}$.
The ET calculation is easy to execute, but it is crucial that ET reactant and product
wavefunctions reflect {\em localized states}. This can be accomplished
using either a fragment guess (shown in the example, see \ref{sec:fragguess}), or a charged atomic
density guess (see \ref{sec:atomscf}). 
For self-exchange ET reactions such as this one, you can use the 
\verb+REORDER+ keyword to move the electron from the first helium to the second (see \ref{sec:vectors}).

Example input :
\begin{verbatim}

#ET reactants:
charge 1
scf 
  doublet; uhf; vectors input fragment HeP.mo He.mo output HeA.mo
# HeP.mo are the vectors for He(+), 
# He.mo  are the vectors for neutral He.
end 
task scf

#ET products:
charge 1
scf 
  doublet; uhf; vectors input HeA.mo reorder 2 1 output HeB.mo
end 
task scf

et
 vectors reactants HeA.mo 
 vectors products HeB.mo
end
task scf et       

\end{verbatim}

Here is what the output looks like for this example:
\begin{verbatim}
                           Electron Transfer Calculation
                           -----------------------------

 MO vectors for reactants: HeA.mo
 MO vectors for products : HeB.mo

 Electronic energy of reactants     H(RR)      -5.3402392824
 Electronic energy of products      H(PP)      -5.3402392824

 Reactants/Products overlap         S(RP)      -0.0006033839

 Reactants/Products interaction energy:
 -------------------------------------
 One-electron contribution         H1(RP)       0.0040314092

 Beginning calculation of 2e contribution
 Two-electron integral screening (tol2e) : 6.03E-11

 Two-electron contribution         H2(RP)      -0.0007837138
 Total interaction energy           H(RP)       0.0032476955

 Electron Transfer Coupling Energy |V(RP)|      0.0000254810
                                                       5.592 cm-1
                                                    0.000693 eV
                                                       0.016 kcal/mol

\end{verbatim}

The overlap between the ET reactant and product states ($S_{RP}$) is small,
so the magnitude of the coupling between the states is also small. 
If the fragment guess
or charged atomic density guess were not used, the Mulliken spin population would be 0.5 on both He atoms, the overlap between
the ET reactant and product states would be \verb+100 %+ and an infinite
$V_{RP}$ would result.
