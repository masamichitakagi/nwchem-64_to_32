%
% $Id$
%
\label{sec:selci}

The selected CI module is integrated into NWChem but as yet no
input module has been written.  The input thus consists of setting the
appropriate variables in the database.

It is assumed that an initial SCF/MCSCF calculation has completed, and
that MO vectors are available.  These will be used to perform a
four-index transformation, if this has not already been performed.

\section{Background}

This is a general spin-adapted, configuration-driven CI program
which can perform arbitrary CI calculations, the only restriction
being that all spin functions are present for each orbital occupation.
CI wavefunctions may be specified using a simple configuration
generation program, but the prime usage is intended to be in
combination with perturbation correction and selection of new
configurations.  The second-order correction (Epstein-Nesbet) to the
CI energy may be computed, and at the same time configurations that
interact greater than a certain threshold with the current CI
wavefunction may be chosen for inclusion in subsequent calculations.
By repeating this process (typically twice is adequate) with the same
threshold until no new configurations are added, the CI expansion may
be made consistent with the selection threshold, enabling tentative 
extrapolation to the full-CI limit.

A typical sequence of calculations is as follows:
\begin{enumerate}
\item Pick as an initial CI reference the previously executed
  SCF/MCSCF.
\item Define an initial selection threshold.
\item Determine the roots of interest in the current reference space.
\item Compute the perturbation correction and select additional
  configurations that interact greater than the current threshold.
\item Repeat steps 3 and 4.
\item Lower the threshold (a factor of 10 is common) and repeat steps
  3, 4, and 5.  The {\em first} pass through step 4 will yield the
  approximately self-consistent CI and CI+PT energies from the {\em
    previous} selection threshold.
\end{enumerate}

To illustrate this, below is some abbreviated output from a
calculation on water in an augmented cc-PVDZ basis set with one frozen
core orbital.  The SCF was converged to high precision in $C_{2v}$
symmetry with the following input
\begin{verbatim}
  start h2o
  geometry; symmetry c2v
    O 0 0 0; H 0 1.43042809 -1.10715266
  end
  basis
    H library aug-cc-pvdz; O library aug-cc-pvdz
  end
  task scf
  scf; thresh 1d-8; end
\end{verbatim}

The following input restarts from the SCF to perform a sequence of
selected CI calculations with the specified tolerances, starting with
the SCF reference.
\begin{verbatim}
  restart h2o
  set fourindex:occ_frozen 1
  set selci:mode select
  set "selci:selection thresholds" \
      0.001 0.001 0.0001 0.0001 0.00001 0.00001 0.000001
  task selci
\end{verbatim}
Table \ref{selcitab} summarizes the output from each of the major
computational steps that were performed.
\begin{table}[htbp]
  \begin{tabular}{c|l|r|l}
       &               & CI        &    \\
  Step &  Description  & dimension &  Energy        \\ \hline
       &               &           &          \\
    1  &  Four-index, one frozen-core &    &      \\
    2  &  Config. generator, SCF default &  1 & \\
    3+4&  CI diagonalization & 1 &  $E_{CI} = -76.041983$ \\
    5  &  PT selection T=0.001 & 1 & $E_{CI+PT} = -76.304797$ \\
    6+7 &  CI diagonalization & 75 & $E_{CI} = -76.110894$ \\
    8  &  PT selection T=0.001& 75 & $E_{CI+PT} = -76.277912$ \\
    9+10&  CI diagonalization & 75 & $E_{CI}(T=0.001) = -76.110894$ \\
    11 &  PT selection T=0.0001 & 75 & $E_{CI+PT}(T=0.001) = -76.277912$ \\
    12+13 & CI diagonalization & 823 & $E_{CI} = -76.228419$ \\
    14 & PT selection T=0.0001 & 823 & $E_{CI+PT} = -76.273751$ \\
    15+16 & CI diagonalization & 841 & $E_{CI}(T=0.0001) = -76.2300544$ \\
    17 & PT selection T=0.00001& 841 & $E_{CI+PT}(T=0.0001) = -76.274073$ \\
    18+19 & CI diagonalization & 2180 &  $E_{CI} = -76.259285$ \\
    20 & PT selection T=0.00001& 2180 & $E_{CI+PT} = -76.276418$ \\
    21+22 & CI diagonalization & 2235 & $E_{CI}(T=0.00001) = -76.259818$ \\
    23 & PT selection T=0.000001 & 2235 & $E_{CI+PT}(T=0.00001) = -76.276478$\\
    24   & CI diagonalization &  11489 & \\ \hline
\end{tabular}
\caption{\label{selcitab} Summary of steps performed in a selected CI
  calculation on water.}
\end{table}

\section{Files}

Currently, no direct control is provided over filenames.  All files
are prefixed with the standard file-prefix, and any files generated by
all nodes are also postfixed with the processor number.  Thus, for
example the molecular integrals file, used only by process zero, might
be called {\tt h2o.moints} whereas the off-diagonal Hamiltonian matrix
element file used by process number eight would be called {\tt
  h2o.hamil.8}.

\sloppy

\begin{description}
\item{\tt ciconf} --- the CI configuration file, which holds
  information about the current CI expansion, indexing vectors, etc.
  This is the most important file and is required for all restarts.
  Note that the CI configuration generator is only run if this file
  does not exist. Referenced only by process zero.
\item{\tt moints} --- the molecular integrals, generated by the four-index
  transformation.  As noted above these must currently be manually
  deleted, or the database entry \verb+selci:moints:force+ set, to
  force regeneration.  Referenced only by process zero.
\item{\tt civecs} --- the CI vectors.    Referenced only by process zero.
\item{\tt wmatrx} --- temporary file used to hold coupling coefficients.
  Deleted at calculation end.  Referenced only by process zero.
\item{\tt rtname, roname} --- restart information for the PT
  selection.  Should be automatically deleted if no restart is
  necessary.  Referenced only by process zero.
\item{\tt hamdg} --- diagonal elements of the Hamiltonian.
  Deleted at calculation end.  Referenced only by process zero.
\item{\tt hamil} --- off-diagonal Hamiltonian matrix elements.  All
  processes generate a file containing a subset of these elements.
  These files can become very large.  Deleted at calculation end.
\end{description}

\fussy

\section{Configuration Generation}

If no configuration is explicitly specified then the previous
SCF/MCSCF wavefunction is used, adjusting for any orbitals frozen in
the four-index transformation.  The four-index transformation must
have completed successfully before this can execute.  Orbital
configurations for use as reference functions may also be explicitly
specified.

Once the default/user-input reference configurations have been
determined additional reference functions may be generated by applying
multiple sets of creation-annihilation operators, permitting for
instance, the ready specification of complete or restricted active
spaces.

Finally, a uniform level of excitation from the current set of
configurations into all orbitals may be applied, enabling, for
instance, the simple creation of single or single+double excitation 
spaces from an MCSCF reference.

\subsection{Specifying the reference occupation}

A single orbital configuration or occupation is specified by
\begin{verbatim}
  ns  (socc(i),i=1,ns)  (docc(i),i=1,nd)
\end{verbatim}
where \verb+ns+ specifies the number of singly occupied orbitals,
\verb+socc()+ is the list of singly occupied orbitals, and
\verb+docc()+ is the list of doubly occupied orbitals (the
number of doubly occupied orbitals, \verb+nd+, is inferred from
\verb+ns+ and the total number of electrons).  All occupations may be
strung together and inserted into the database as a single integer
array with name \verb+"selci:conf"+.  For example, the input
\begin{verbatim}
  set "selci:conf" \
    0                1  2  3  4 \
    0                1  2  3 27 \
    0                1  3  4 19 \
    2   11 19        1  3  4 \
    2    8 27        1  2  3 \
    0                1  2  4 25 \
    4   3  4 25 27   1  2 \
    4   2  3 19 20   1 4 \
    4   2  4 20 23   1 3  
\end{verbatim}
specifies the following nine orbital configurations
\begin{verbatim}
   1(2)  2(2)  3(2)  4(2)
   1(2)  2(2)  3(2) 27(2)
   1(2)  3(2)  4(2) 19(2)
   1(2)  3(2)  4(2) 11(1) 19(1)
   1(2)  2(2)  3(2)  8(1) 27(1)
   1(2)  2(2)  4(2) 25(2)
   1(2)  2(2)  3(1)  4(1) 25(1) 27(1)
   1(2)  2(1)  3(1)  4(2) 19(1) 20(1)
   1(2)  2(1)  3(2)  4(1) 20(1) 23(1)
\end{verbatim}
The optional formatting of the input is just to make this arcane
notation easier to read.  Relatively few configurations can be
currently specified in this fashion because of the input line limit of
1024 characters.

\subsection{Applying creation-annihilation operators}

Up to 10 sets of creation-annihilation operator pairs may be
specified, each set containing up to 255 pairs.  This suffices to
specify complete active spaces with up to ten electrons.

The number of sets is specified as follows,
\begin{verbatim}
  set selci:ngen 4
\end{verbatim}
which indicates that there will be four sets.  Each set is then
specified as a separate integer array
\begin{verbatim}
  set "selci:refgen  1" 5 4    6 4   5 3   6 3  
  set "selci:refgen  2" 5 4    6 4   5 3   6 3  
  set "selci:refgen  3" 5 4    6 4   5 3   6 3  
  set "selci:refgen  4" 5 4    6 4   5 3   6 3  
\end{verbatim}
In the absence of friendly, input note that the names
\verb+"selci:refgen n"+ must be formatted with n in \verb+I2+
format. Each set specifies a list of creation-annihilation operator
pairs (in that order).  So for instance, in the above example each set
is the same and causes the excitations
\begin{verbatim}
  4->5   4->6   3->5   3->6
\end{verbatim}
If orbitals 3 and 4 were initially doubly occupied, and orbitals 5 and
6 initially unoccupied, then the application of this set of operators
four times in succession is sufficient to generate the four electron
in four orbital complete active space.

The precise sequence in which operators are applied is
\begin{enumerate}
\item loop through sets of operators
\item loop through reference configurations
\item loop through operators in the set
\item apply the operator to the configuration, if the result is new add it
  to the new list
\item end the loop over operators
\item end the loop over reference configurations
\item augment the configuration list with the new list
\item end the loop over sets of operators
\end{enumerate}

\subsection{Uniform excitation level}

By default no excitation is applied to the reference configurations.
If, for instance, you wanted to generate a single excitation CI space
from the current configuration list, specify
\begin{verbatim}
set selci:exci 1
\end{verbatim}
Any excitation level may be applied, but since the list of
configurations is explicitly generated, as is the CI Hamiltonian
matrix, you will run out of disk space if you attempt to use more than
a few tens of thousands of configurations.

\section{Number of roots}

By default, only one root is generated in the CI diagonalization or
perturbation selection.  The following requests that 2 roots be
generated
\begin{verbatim}
  set selci:nroot 2
\end{verbatim}
There is no imposed upper limit.  If many roots are required, then, to
minimize root skipping problems, it helps to perform an initial
approximate diagonalization with several more roots than required, and
then resetting this parameter once satisfied that the desired
states are obtained.

\section{Accuracy of diagonalization}

By default, the CI wavefunctions are converged to a residual norm of
$10^{-6}$ which provides similar accuracy in the perturbation
corrections to the energy, and much higher accuracy in the CI
eigenvalues.  This may be adjusted with
\begin{verbatim}
 set "selci:diag tol" 1d-3
\end{verbatim}
the example setting much lower precision, appropriate for the
approximate diagonalization discussed in the preceding section.

\section{Selection thresholds}

When running in the selected-CI mode the program will loop
through a list of selection thresholds ($T$), performing the CI
diagonalization, computing the perturbation correction, and augmenting
the CI expansion with configurations that make an energy lowering to
any root greater than $T$.  The list of selection thresholds is
specified as follows
\begin{verbatim}
  set "selci:selection thresholds" \
      0.001 0.001 0.0001 0.0001 0.00001 0.00001 0.000001
\end{verbatim}

There is no default for this parameter.


\section{Mode}

By default the program runs in \verb="ci+davids"= mode and just
determines the CI eigenvectors/values in the current configuration
space.  To perform a selected-CI with perturbation correction use the
following
\begin{verbatim}
  set selci:mode select
\end{verbatim}
and remember to define the selection thresholds.

\section{Memory requirements}

No global arrays are used inside the selected-CI, though the
four-index transformation can be automatically invoked and it does use
GAs.  The selected CI replicates inside each process
\begin{itemize}
\item all unique two-electron integrals in the MO basis that are
  non-zero by symmetry, and
\item all CI information, including the CI vectors.
\end{itemize}
These large data structures are allocated on the local stack.  A fatal
error will result if insufficient memory is available.

\section{Forcing regeneration of the MO integrals}

When scanning a potential energy surface or optimizing a geometry the
MO integrals need to be regenerated each time.  Specify
\begin{verbatim}
  set selci:moints:force logical .true.
\end{verbatim}
to accomplish this.

\section{Disabling update of the configuration list}

When computing CI+PT energy the reference configuration list is
normally updated to reflect all configurations that interact more than
the specified threshold.  This is usually desirable.  But when
scanning a potential energy surface or optimizing a geometry the
reference list must be kept fixed to keep the potential energy surface
continuous and well defined.  To do this specify
\begin{verbatim}
  set selci:update logical .false.
\end{verbatim}

\section{Orbital locking in CI geometry optimization}

The selected CI wavefunction is not invariant to orbital rotations or 
to swapping two or more orbitals. Orbitals could be swapped or rotated
when the geometry is changed in a geometry optimization step. The keyword 
\verb#lock# has to be set in the SCF/MCSCF (vectors) input block to keep the
orbitals in the same order throughout the geometry optimization.
